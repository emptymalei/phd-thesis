%!TEX root = ../dissertation.tex
\chapter{\label{chap:conclusion}Conclusion}


The neutrino is a common product in astrophysical environments which makes neutrino physics a key ingredient of astrophysics. Since different neutrino flavors have different impact on the dynamics of astrophysical objects, identifying the flavor dynamics of the neutrino becomes important. One the other hand, the neutrino converts between different flavors while it propagates due to the mismatch of flavor states and mass states. Thus the neutrino doesn't maintain its flavor as produced. Even more, the conversions are altered by neutrino interactions with the matter or the neutrino itself. 

The interactions with matter lead to the so called MSW effect where neutrinos may experience maximum flavor conversion for some specific matter density. In this dissertation, I have also explained the neutrino parametric oscillations where the varying matter profile serves as the driving field for the neutrino two-level quantum system. The matter profile drives the neutrino oscillations to resonance when the frequency of the driving potential matches the energy gap of the two energy levels, which is quite similar to Rabi oscillations. However, the matter potential provides more than one driving frequencies. Those different driving frequencies interfere with each other, which indicates that a second driving potential might lead to destruction of resonances of the original driving potential. This has been proved in the dissertation. I also prove that the interference might also enhance the resonance. The Rabi oscillations point of view has provided a simple picture to understand the neutrino oscillations in oscillatory matter profile.

For future research, neutrino oscillations in more realistic matter profile has to be calculated to locate the possible resonances. For example, the matter density in supernovae is Kolmogorov-like. To understand the matter effect on neutrino oscillations in supernovae, the possible resonances can be identified using the resonance condition and the interference effect. As for applications of this parametric resonance, one of them is neutrino tomography. Using resonances of neutrino oscillations at different energies, it is possible to infer the interior structure of celestial objects.

As neutrino oscillations with self-interactions, the equation of motion is linearized for more analysis in this dissertation. I. Izaguirre et al. has concluded that the gaps of the dispersion relations corresponds to flavor instabilities.
