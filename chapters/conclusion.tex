%!TEX root = ../dissertation.tex
\chapter{\label{chap:conclusion}Conclusions and Future Work}

Neutrinos are abundantly produced in astrophysical environments such as core-collapse supernovae and binary neutron star mergers. They play important roles in the chemical and physical evolutions of these environments. Since different neutrino flavors have different influence on the environments, identifying the neutrino fluxes in different flavors becomes important. Meanwhile, neutrinos oscillate between different flavors while they propagate through space due to the mismatch of the flavor states and the mass states. Therefore it is important to understand the flavor evolution of dense neutrino media.

The interactions between the neutrino and matter can cause the neutrino to experience flavor transformation through the so called MSW mechanism if the matter profile is smooth. Neutrinos can also experience parametric resonances when there exist oscillatory perturbations to a smooth profile of the matter density. In this dissertation, I showed that this parametric resonance behavior can be understood as Rabi oscillations which provides a simple yet intuitive picture to understand this interesting phenomenon. Interestingly, there exists an infinite number of Rabi modes even for a sinusoidal matter perturbation. I showed that most of these Rabi modes are irrelevant in a real physical system because of their tiny amplitudes, which is also true for multi-frequency matter profiles. I also derived a criterion which can be used to determine whether an off-resonance Rabi mode can significantly alter the behavior of the Rabi oscillation.

For future research, neutrino oscillations can be calculated with realistic matter profiles in supernovae or stars. The criteria I have derived in my dissertation can be used to select the on-resonance and off-resonance Rabi modes that are important to neutrino flavor transformation and to identify the parametric resonances in these environments.
% To understand the matter effect on neutrino oscillations in supernovae, the possible resonances can be identified using the resonance condition and the interference effect. As for applications of this parametric resonance, one of them is neutrino tomography. Using resonances of neutrino oscillations at different energies, it is possible to infer the interior structure of celestial objects.

A dense neutrino medium can experience collective flavor oscillations because of the neutrino self-interaction. The linearized flavor stability analysis and the dispersion relations of the collective modes have been used to identify the physical regimes where collective oscillations may occur. In this dissertation, I applied these methods to the neutrino media with discrete and continuous angular emissions. I showed that, contrary to the conjecture by I. Izaguirre et al. in Ref.~\cite{Izaguirre2016a}, the flavor instabilities of a neutrino medium are not necessarily associated with the gaps in the dispersion relation curves of the collective modes. More work needs to be done to understand the origin and nature of the flavor instabilities of a dense medium and their relation to the dispersion relations of the collective modes.

% Recently, F. Capozzi has developed more theories to use the dispersion relations to understand flavor instabilities~\cite{Capozzi2017}. They have found criteria for four different types of instabilities. The two important instabilities are absolute flavor instability and convective flavor instability, where the neutrino flavor perturbation grows everywhere and the neutrino flavor perturbation decays locally but grows in other locations, respectively. However, the theories has to be verified for multiple emission angles.

The problem of neutrino oscillations in supernovae is further complicated by the presence of the neutrino halo formed by scattered neutrino fluxes. In this dissertation, I have developed a toy model to explore neutrino oscillations in the presence of scattered neutrino fluxes. I have also developed a relaxation method and a parallel numerical code based on this algorithm to solve this toy model. More calculations need to be done to find out how the scattered neutrino fluxes may affect neutrino oscillations.

% The so called neutrino halo problem is also discussed in this dissertation. The problem arises since the supernova neutrinos are scattered when they are propagating. The backward propagating neutrinos encounters the forward propagating neutrinos and interact with them. The problem becomes mathematically and numerically difficult since it is a nonlocal boundary value problem. In this dissertation, I have designed and tested the relaxation method to solve the halo problem numerically. The relaxation method starts with some initial configuration of neutrino states in space and allow the neutrinos to relax to an equilibrium configuration. To efficiently solve the problem, the code utilizes the power of parallel computing. I also found that the single beam line model, where neutrinos are emitted on a line all in one direction and reflected at the same distance, has similar dynamics as the bipolar model. This similarity makes it possible to solve the linear regime of the problem and validate the numerical.

% The neutrino halo problem brings in a lot of variables to the neutrino oscillations around supernovae. Since the relaxation method has been proven to work, more variable models can be solved numerically and understand the effect of the neutrino halo. The symmetries in the halo problem should be broken to investigate the possible new flavor instabilities.
