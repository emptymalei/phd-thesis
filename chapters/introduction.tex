%!TEX root = ../phd-thesis-lei-ma.tex
%!TeX spellcheck = en-US
\chapter{Introduction}
\label{introduction}

The neutrino has been one of the most astonishing particles in history. Its glorious history started with beta decay, i.e., the emission of electrons in nuclear decays, such as
\begin{equation*}
{}^A_Z \mathrm X \to {}_{Z+1}^A\mathrm X + e^- +\bar \nu_e .
% {}^{14}_{6} \mathrm C \to {}^{17}_{7}\mathrm N + \mathrm e^{-} + \bar\nu_{\mathrm e}.
\end{equation*}
The fact that the electron energy spectrum in beta decay is continuous indicates the existence of a third product other than ${}^{A}_{Z+1}\mathrm X$ and $\mathrm e^-$. It was then proven to be anti-neutrinos. In such reactions, charged current weak interaction converts a down quark in a neutron to an up quark while releasing electrons and an anti-electron neutrino,
\begin{equation}
n\to p + e^- + \bar \nu_e .
\end{equation}
More generally, positron emission and electron capture are also neutrino-related nuclear reactions which is explained in Table.~\ref{table:Neutrino_Reactions}. There are three different flavors of neutrinos, namely electron flavor, muon flavor, and tau flavor as shown in Table.~\ref{table:neutrino-properties}. The direct detection of neutrinos was done two decades later, by Clyde Cowan and Frederick Reines in 1956~\cite{Cowan1956}. The Cowan-Reines experiment used nuclear reactor neutrinos as source. As the detection of neutrinos becomes feasible, Ray Davis and John Bahcall et al worked out the solar neutrino flux and led the Homestake experiment to measure the solar neutrinos. The results revealed that the neutrino flux detected was less than the prediction by solar models, which is the well-known solar neutrino problem~\cite{Bahcall1973}. It is known today that the solution to the problem is related to neutrinos. Electron neutrinos produced in the solar core convert to other flavors as they propagate, which is referred to as neutrino oscillations.



\begin{table}[ht]
\centering
 \begin{tabular}{|c | c | c|}
 \hline
 Reaction & Equation & Boson   \\ [0.5ex]
 \hline
 Electron emission & ${}^A_Z X \to {}^A_{Z+1}X + e^- +\bar \nu_e$ & $W$  \\
 Positron emission & ${}^A_Z X \to {}^A_{Z-1}X + e^+ + \nu_e$ & $W$  \\
 Electron capture & ${}^A_Z X + e^- \to {}^A_{Z-1}X  + \nu_e$ &  $W$ \\
 Positron capture & ${}^A_Z X + e^+ \to {}^A_{Z+1}X  + \bar\nu_e$ &  $W$ \\
 [0.5ex]
 \hline

 Electron annihilation &  $e^- + e^+  \to \nu + \bar\nu $  & $W$, $Z$ \\
%  Electron annihilation &  $e^- + e^+  \to \nu + \bar\nu $  & $Z$ \\
 Bremsstrahlung & $X+X \to X + X + \nu + \bar\nu$ & $W$, $Z$ \\
 [0.5ex]
 \hline

  Neutrino capture & ${}^A_{Z}X + \overset{(-)}{\nu_e} \to {}^A_{Z\mp 1}X + e^\pm $ & W\\
  [1ex]
 \hline
 $e^-\nu$ scattering & $e^- + \overset{(-)}{\nu_e} \to e^- + \overset{(-)}{\nu_e} $ &  $W$ \\
 $e^-\nu$ scattering & $e^{\pm} + \overset{(-)}{\nu_e} \to e^{\pm} + \overset{(-)}{\nu_e} $ &  $Z$ \\
 Nucleon scattering & $ {}^A_Z X + \overset{(-)}{\nu} \to {}^A_Z X + \overset{(-)}{\nu} $ &  Z\\
 [0.5ex]
 \hline
 \end{tabular}
 \caption{Neutrino related nuclear or leptonic reactions}
\label{table:Neutrino_Reactions}
\end{table}


Pontecorvo proposed that neutrinos change flavors while they propagate~\cite{Pontecorvo1968}. The field of neutrino oscillations has grown significantly into a broad field in physics since then. Apart from particle physics and solar models, the interest in neutrino oscillations has been expanded to the field of core-collapse supernova explosions and accretion discs since neutrinos also participate in nuclear reaction chains in stars, synthesis of heavy and rare elements and more. For instance, heavy stars explode when nuclear reactions fail to provide enough pressure to sustain the star against gravity and become core-collapse supernovae. During the collapse, the inner core is compressed to almost nuclear density, which provides a stiff equation of state. Materials in-falling onto the stiff core are bounced outward plowing through the inward flow, so that shock waves are formed. Supernova simulations show that the shock wave itself is not always energetic enough to trigger explosions for core-collapse supernovae \cite{Janka2016b}. To revive the shock, energy has to be deposited. The most prominent solution is to introduce reheating of the shock wave by neutrinos~\cite{Janka2016b}. In principle, to impose neutrino driven mechanism into computer simulations of supernova, the flux and flavor content of neutrinos have to be known everywhere. Thus neutrino oscillations in dense shock material and dense neutrino background become the key to the supernova explosion problem. Observation-wise, neutrino signals are crucial for validation of our models about stellar evolution. In fact, detection of galactic core-collapse supernova neutrinos is on the task list of the Deep Underground Neutrino Experiment (DUNE)~\cite{Kemp2017}.

\begin{table}[ht]
\centering
 \begin{tabular}{|c | c | c|}
%  \hline
%  Property & Equation & Boson   \\ [0.5ex]
 \hline
  Electric Charge & 0\\
  \hline
  Spin & $1/2$ \\
\hline
 Mass & $<2\mathrm{eV}$  \\
 \hline
 Interactions & Weak, Gravitation  \\
 \hline
 Flavors & $\nu_e$, $\nu_\mu$, $\nu_\tau$ \\
 \hline
 Chirality & Left \\
 \hline
 Hypercharge & $-1$ \\
 \hline

 \end{tabular}
 \caption{Properties of Neutrinos~\cite{Patrignani:2016xqp}}
\label{table:neutrino-properties}
\end{table}


As we have seen, it is crucial to understand neutrino flavors. In Chapter~\ref{chap:basics} I will review neutrino oscillations in vacuum, with flavor-isospin picture demonstrated. Meanwhile, neutrino oscillations are ingredients of many other astrophysical, cosmological, and astronomical problems, such as neutron star mergers, dark matter, nucleosynthesis, etc. In order to gain a better understanding of neutrinos in these exotic environments, neutrino oscillations in dense matter background and dense neutrino background have to be thoroughly investigated. The seminal work by Mikheev--Smirnov--Wolfenstein proved neutrino interactions with matter background have significant effect on neutrino oscillations. They showed that neutrinos propagating through decreasing matter density experience a potential that alters the flavor conversions (MSW effect), which may also lead to maximum conversions between flavors~\cite{Mikheev:1986gs,wolf78,wolfensteinprd1979}. It is also know that neutrino oscillations in more general matter density profiles exhibit interesting phenomena. Resonances are found as the characteristic length scale in matter density profile and characteristic length scale of the neutrinos satisfies certain relations. I will discuss in details on neutrino oscillations in arbitrary matter density profiles, which is decomposed into Fourier modes and interpreted as superposition of Rabi oscillations in chapter~\ref{chap:matter}. Apart from dense matter background, neutrinos also interact with neutrinos themselves and introducing nonlinear dynamics. The neutrino self-interactions are analyzed using linear stability analysis. In chapter~\ref{chap:dr}, I will review how neutrino self-interactions change neutrino oscillations, as well as the dispersion relations in linear stability analysis. I will also discuss the neutrino halo problem. The halo problem exists because neutrino propagating out of dense matter medium will be scattered and forming a neutrino halo. Some of the neutrinos will propagate backward and interact with forward propagating neutrinos and alter the neutrino flavors. Mathematically speaking, the neutrino halo problem is a nonlocal boundary value problem. I will explain the numerical relaxation scheme that we developed, which we have proven to be a promising method to solve neutrino halo problem. Chapter~\ref{chap:conclusion} summarizes my work and discusses the future explorations of the field.
