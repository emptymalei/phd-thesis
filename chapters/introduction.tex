%!TEX root = ../phd-thesis-lei-ma.tex
%!TeX spellcheck = en-US
\chapter{Introduction}
\label{introduction}

The neutrino has been one of the most astonishing particles in history. Its glorious history started with the observation of beta decay, i.e., the emission of electrons in nuclear decays, such as
\begin{equation*}
{}^A_Z \mathrm X \to {}_{Z+1}^A\mathrm X + e^- +\bar \nu_e .
% {}^{14}_{6} \mathrm C \to {}^{17}_{7}\mathrm N + \mathrm e^{-} + \bar\nu_{\mathrm e}.
\end{equation*}
The fact that the electron energy spectrum in the beta decay process is continuous indicates the existence of a third product other than ${}^{A}_{Z+1}\mathrm X$ and $\mathrm e^-$. It was then proven to be an anti-neutrino. In the above reactions, the charged current weak interaction converts a down quark in the neutron to an up quark while releasing an electron and an anti-electron neutrino,
\begin{equation}
n\to p + e^- + \bar \nu_e .
\end{equation}
More generally, positron/electron emission and positron/electron capture processes are also neutrino-related nuclear reactions which are listed in Table~\ref{table:Neutrino_Reactions}. There are three different flavors of neutrinos, namely the electron flavor, the muon flavor, and the tau flavor as shown in Table~\ref{table:neutrino-properties}. The direct detection of neutrinos was done by Clyde Cowan and Frederick Reines in 1956~\cite{Cowan1956}. The Cowan-Reines experiment used nuclear reactor neutrinos as its source.

As the detection of neutrinos becomes feasible, Ray Davis and John Bahcall et al worked out the solar neutrino flux and led the Homestake experiment to measure the solar neutrinos. The results revealed that the neutrino flux detected was less than what is predicted by solar models, which is the well-known solar neutrino problem~\cite{Bahcall1973}. It is known today that the solution to the problem is related to the neutrino. Electron neutrinos produced in the solar core transform to other flavors as they propagate, which is referred to as neutrino oscillations. The theory of neutrino oscillations was first proposed by Pontecorvo in 1968~\cite{Pontecorvo1968}. The field of neutrino oscillations has grown significantly into a broad field in physics since then.

\begin{table}[ht]
\centering
 \begin{tabular}{|c | c | c|}
 \hline
 Reaction Type & Process & Mediator(s)   \\ [0.5ex]
 \hline
 Electron emission & ${}^A_Z X \to {}^A_{Z+1}X' + e^- +\bar \nu_e$ & $W^{\pm}$  \\
 Positron emission & ${}^A_Z X \to {}^A_{Z-1}X' + e^+ + \nu_e$ & $W^{\pm}$  \\
 Electron capture & ${}^A_Z X + e^- \to {}^A_{Z-1}X  + \nu_e$ &  $W^{\pm}$ \\
 Positron capture & ${}^A_Z X + e^+ \to {}^A_{Z+1}X  + \bar\nu_e$ &  $W^{\pm}$ \\
 [0.5ex]
 \hline

 $e^{\pm}$ annihilation &  $e^- + e^+  \to \nu + \bar\nu $  & $W^{\pm}$, $Z$ \\
%  Electron annihilation &  $e^- + e^+  \to \nu + \bar\nu $  & $Z$ \\
 Bremsstrahlung & $X+X' \to X + X' + \nu + \bar\nu$ & $Z$ \\
 [0.5ex]
 \hline

  $\nu (\bar\nu)$ capture & ${}^A_{Z}X + \overset{(-)}{\nu_e} \to {}^A_{Z\mp 1}X + e^\pm $ & $W^{\pm}$\\
  [1ex]
 \hline
 $e^\pm\nu$ scattering & $e^- + \overset{(-)}{\nu_e} \to e^- + \overset{(-)}{\nu_e} $ &  $W^{\pm}$, $Z$ \\
 % $e^\pm\nu$ scattering & $e^{\pm} + \overset{(-)}{\nu_e} \to e^{\pm} + \overset{(-)}{\nu_e} $ &  $Z$ \\
 Nucleon scattering & $ {}^A_Z X + \overset{(-)}{\nu} \to {}^A_Z X + \overset{(-)}{\nu} $ &  Z\\
 [0.5ex]
 \hline
 \end{tabular}
 \caption{Neutrino related nuclear and leptonic reactions.}
\label{table:Neutrino_Reactions}
\end{table}


Apart from particle physics and solar models, the interest in neutrino oscillations has been expanded to the fields of core-collapse supernova explosions and accretion discs since neutrinos also participate in nuclear reactions in stars, synthesis of heavy and rare elements, and more. For instance, when the core of a massive star run out of nuclear fuel, it collapses, explodes, and becomes a core-collapse supernova. During the collapse, the inner core is compressed to almost nuclear density, which has a stiff equation of state. The material falling onto the stiff core are bounced outward which forms a shock wave plowing through the inward flow. Supernova simulations show that the shock wave itself is not always energetic enough to produce the explosion~\cite{Janka2016b}. To revive the shock, more energy has to be deposited behind the shock. A possible solution is to introduce reheating of the shock by neutrinos~\cite{Janka2016b}. In order to implement neutrino-driven mechanism in computer simulations of supernovae, the flux and flavor content of neutrinos have to be known everywhere. Thus neutrino oscillations in dense matter become the a key to the supernova explosion problem. Observation-wise, neutrino signals are crucial for validation of our models for supernovae. In fact, detection of galactic core-collapse supernova neutrinos is on the task list of the Deep Underground Neutrino Experiment (DUNE)~\cite{Kemp2017}.

\begin{table}[ht]
\centering
 \begin{tabular}{|c | c | c|}
%  \hline
%  Property & Equation & Boson   \\ [0.5ex]
 \hline
  Electric Charge & 0\\
  \hline
  Spin & $1/2$ \\
\hline
 Mass & $<2~\mathrm{eV}$ \\
 \hline
 Interactions & Weak, Gravitation  \\
 \hline
 Flavors & $\nu_e$, $\nu_\mu$, $\nu_\tau$ \\
 \hline
 Chirality & Left \\
 \hline
 Hypercharge & $-1$ \\
 \hline

 \end{tabular}
 \caption{The physical properties of the neutrino~\cite{Patrignani:2016xqp}.}
\label{table:neutrino-properties}
\end{table}


% As we have seen, it is crucial to understand neutrino flavors.
The rest of the disseration is organized as the follows.
In Chapter~\ref{chap:basics} I will review neutrino oscillations in vacuum and explain the flavor-isospin picture.
% Meanwhile, neutrino oscillations are ingredients of many other astrophysical, cosmological, and astronomical problems, such as neutron star mergers, dark matter, nucleosynthesis, etc. In order to gain a better understanding of neutrinos in these exotic environments, neutrino oscillations in dense matter background and dense neutrino background have to be thoroughly investigated. The seminal work by Mikheev--Smirnov--Wolfenstein proved neutrino interactions with matter background have significant effect on neutrino oscillations. They showed that neutrinos propagating through decreasing matter density experience a potential that alters the flavor conversions (MSW effect), which may also lead to maximum conversions between flavors~\cite{Mikheev:1986gs,wolf78,wolfensteinprd1979}. It is also know that neutrino oscillations in more general matter density profiles exhibit interesting phenomena. Resonances are found as the characteristic length scale in matter density profile and characteristic length scale of the neutrinos satisfies certain relations.
In chapter~\ref{chap:matter}, I will discuss in details neutrino oscillations in arbitrary matter density profiles, which is decomposed into Fourier modes and interpreted as a superposition of Rabi oscillations.
%Apart from dense matter background, neutrinos also interact with neutrinos themselves and introducing nonlinear dynamics. The neutrino self-interactions are analyzed using linear stability analysis.
In chapter~\ref{chap:dr}, I will review how neutrino self-interactions change neutrino oscillations when a significant neutrino backgroun is present such as core-collapse supernovae
%, as well as the dispersion relations in linear stability analysis. I will also discuss the neutrino halo problem. The halo problem exists because neutrino propagating out of dense matter medium will be scattered and forming a neutrino halo. Some of the neutrinos will propagate backward and interact with forward propagating neutrinos and alter the neutrino flavors. Mathematically speaking, the neutrino halo problem is a nonlocal boundary value problem. I will explain the numerical relaxation scheme that we developed, which we have proven to be a promising method to solve neutrino halo problem.
In chapter~\ref{chap:conclusion}, I will summarize my work and discusses the future explorations of the field.
