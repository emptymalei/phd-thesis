%!TEX root = ../dissertation.tex
\chapter{Introduction}
\label{introduction}

Neutrino has been one of the most astonishing particles in history. It's glorious history started with beta decay, which is the emission of electrons in nuclear decays, such as
\begin{equation*}
{}^A_Z \mathrm X \to {}_{Z+1}^A\mathrm X + e^- +\bar \nu_e .
% {}^{14}_{6} \mathrm C \to {}^{17}_{7}\mathrm N + \mathrm e^{-} + \bar\nu_{\mathrm e}.
\end{equation*}
The fact that electron energy spectrum in beta decay is continuous indicates the existence of a third product other than ${}^{A}_{Z}\mathrm X$ and $\mathrm e^-$. It was proven to be anti-neutrinos. Neutrinos are fermions with three different flavors, namely electron flavor, muon flavor, and tau flavor. The direct detection of neutrinos was done two decades later, by Clyde Cowan and Frederick Reines in 1956~\cite{Cowan1956}. The Cowan-Reines experiment used nuclear reactor neutrinos as source. As the detection of neutrinos becomes feasible, Ray Davis and John Bahcall et al worked out the solar neutrino flux and lead the Homestake experiment to measure the solar neutrinos. The results revealed that the neutrino flux detected was less than the prediction by solar models, which is the well-known solar neutrino problem~\cite{Bahcall1973}. It's know today that the solution to the problem is related to neutrinos. Electron neutrinos produced in the solar core convert to other flavors as they propagate, which is dubbed as neutrino oscillations. 

Pontecorvo proposed that neutrinos change flavors while they propagate~\cite{Pontecorvo1968}. The field of neutrino oscillations has grown significantly into a broad field in physics since then. Apart from particle theories and solar models, the interest in neutrino oscillations has been expanded to the field of core-collapse supernova explosions and accretion discs. Heavy stars explode when nuclear reactions fail to provide enough pressure to maintain current radius and become core-collapse supernovae. During the collapse, the inner core is compressed to almost nuclear density, which provides a stiff equation of state. Materials in-falling onto the stiff core are bounced outward plowing through the inward flow, so that shock waves are formed. Supernova simulations show that the shock wave itself is not always energetic enough to trigger explosions for core-collapse supernovae \cite{Janka2016b}. To revive the shock, energy has to be deposited. The most prominent solution is to introduce reheating of the shock wave by neutrinos~\cite{Janka2016b}. In principle, to impose neutrino driven mechanism to computer simulations of supernova, the flux and flavor content of neutrinos has to be well known everywhere. Thus neutrino oscillations in dense shock material and dense neutrino background become the key to supernova explosion problem. Observation-wise, neutrino signals are crucial for validation of our models about stellar evolution. In fact, detection of galactic core-collapse supernova neutrinos is on the task list of DUNE (Deep Underground Neutrino Experiment)~\cite{Kemp2017}.

Moreover, neutrino oscillations are ingredients of many other astrophysical, cosmological, and astronomical problems, such as neutron star mergers, dark matter, nucleosynthesis, etc. In order to gain better understanding of neutrinos in these exotic environments, neutrino oscillations in matter background and dense neutrino background have to be thoroughly investigated. The seminal work by Mikheev--Smirnove--Wolfenstein proved neutrino interactions with matter background has significant effect on neutrino oscillations. They showed that neutrinos propagate through decreasing matter density experience a potential that alters the flavor conversions (MSW effect), which may also lead to maximum conversions between flavors~\cite{Mikheev:1986gs,wolf78,wolfensteinprd1979}. Chapter~\ref{chap:basics} will be a review of neutrino oscillations in vacuum, matter background, and dense neutrino background. However, neutrino oscillations in more general matter density profiles. Resonances occur as some length scales in matter density profile and characteristic length scale of the problem satisfies some certain relations. I will discuss in details on neutrino oscillations in arbitrary matter density profiles, which is interpreted as superposition of Rabi oscillations in chapter~\ref{chap:matter}. Apart from dense matter background, neutrinos also interact with neutrinos themselves and introducing nonlinear dynamics. The neutrino self-interactions are analyzed using linear stability analysis. In chapter~\ref{chap:dr}, I will review how neutrino self-interactions change neutrino oscillations, as well as the dispersion relations in linear stability analysis. Finally, chapter~\ref{chap:halo} will visit the problem of neutrino halo problem. Neutrino propagating out of dense matter medium will be scattered and forming a neutrino halo. Some of the neutrinos will propagate backward and interact with forward propagating neutrinos and alter the neutrino flavors. Neutrino halo problem is more like a nonlocal boundary value problem. I will explain the numerical relaxation scheme that we developed which is proven to work.


