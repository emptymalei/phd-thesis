%!TEX root = ../phd-thesis-lei-ma.tex
%
\begin{abstract}%

Neutrinos are abundantly produced in astrophysical environments such as core-collapse supernovae and binary neutron star mergers. These astrophysical objects are surrounded by dense media. Neutrino flavor conversions in the dense media play important roles in the physical and chemical evolutions of the environments. In this dissertation, I study two mechanisms through which neutrinos may change their flavors.

In the first mechanism, neutrinos can experience flavor conversions through interactions with oscillatory perturbations in matter distributions. I show that this mechanism can be understood intuitively as Rabi oscillations. I also derive criteria which can be used to determine whether such parametric resonances exist in a given environment.

% One of the interesting and important problems in astrophysics is the mechanism of core-collapse supernova explosions. Many numerical simulations have shown that the explosion shock would stall. Different proposals have been made to explain the core-collapse supernovae, among which the neutrino mechanism is promising and most researched one. To explore the mechanism, prediction of the neutrino flavors in core-collapse supernovae is crucial. Neutrino flavor conversions are altered by the matter, neutrinos themselves, as well as other factors such as the geometries of the neutrino emissions. The complexity of the problem requires breaking it down into investigations of each simple yet specific situation.

% Neutrinos propagating through a matter background experience a potential which changes the flavor conversions. One of the important mechanisms is the Mikheyev–Smirnov–Wolfenstein effect. However, much more complicated density profiles of matter, such as periodic density profiles, may lead to large flavor conversion, which is dubbed as stimulated oscillations by J. Kneller et al. Mathematics of such large conversion has been established but without clear pictures. For the two-flavor scenario, neutrino oscillations is a two-level quantum system, and it reminds us of many two-level quantum problems that have been solved in the past. We draw analogies between neutrinos passing through matter and Rabi oscillations in optics, which allows us to calculate resonance conditions and flavor survival probability easily.

In the second mechanism, the whole neutrino medium can experience flavor conversions because of the neutrino self-interactions. Applying the linearized flavor stability analysis method to the dense neutrino medium with discrete and continuous angular emissions, I show that, contrary to a recent conjecture by I. Izaguirre et al., the flavor instabilities are not always associated with the gaps in dispersion relation curves of the collective modes of neutrino oscillations. I also develop a toy model and a numerical scheme which can be used to explore neutrino oscillations in an environment where scattered neutrino fluxes are present.

% As for neutrinos flavors with high number densities, nonlinear interactions come into play since neutrino forward scattering provides another potential that is related to the flavor of the neutrinos themselves. Nonlinearity makes the flavor conversion hard to predict by intuition. The treatment is linearizing the equation of motion and identifying instabilities. One of the tricks in the realm is to utilize the dispersion relation. In principle, dispersion relations tell us how waves propagate for different wave numbers and frequencies. However, the neutrino problem is much more complicated. Situations that are inconsistent with the dispersion relation approach are identified.

% Finally, forward scattering of supernova neutrinos are not the only thing that happens. During propagation around a supernova, neutrinos may be scattered in every direction, which forms a neutrino halo. The halo couples the neutrinos nonlocally, which then becomes a nonlocal boundary value problem. One of the solutions is the relaxation method. Starting from some state of neutrinos and relaxing the system into equilibrium has proven to be a working algorithm. A numerical algorithm is developed and neutrino line model with back scattering is investigated.

\end{abstract}
\clearpage %(required for 1-page abstract)
