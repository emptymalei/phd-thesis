%!TEX root = ../phd-thesis-lei-ma.tex
\begin{abstract}

  One of the interesting and import problem in astrophysics is the mechanism of core-collapse supernova explosions. Many of numerical simulations has shown that the explosion shock would stall. Different proposals has been made to explain the core-collapse supernovae, among which the neutrino mechanism is the promising and most researched one. To explore the mechanism, prediction of the neutrino flavors in core-collapse supernovae is crucial. Neutrino flavor conversions are altered by the matter, neutrino themselves, as well as other factors such as the geometries of the neutrino emissions. The complicity of the problem requires breaking it down into investigations of each simple yet specific situations.

   Neutrinos propagating through matter background experience a potential which changes the flavor conversions. One of the important mechanism is the Mikheyev–Smirnov–Wolfenstein effect. However, much more complicated density profiles of matter, such as periodic density profiles, may lead to large flavor conversion, which is dubbed as stimulated oscillations by J. Kneller et al. Mathematics of such large conversion has been established but without clear pictures. For two flavor scenario, neutrino oscillations is a quantum two level systems, and it reminds us of many quantum two level problem that has been solved in the past. We draw analogies between neutrinos passing through matter and Rabi oscillations in optics which allows us to calculate resonance conditions and flavor survival probability easily.

   As for neutrinos flavors with high number densities, nonlinear interactions come into play since neutrino forward scattering provides another potential that is related to the flavor of the neutrinos themselves. Nonlinearity makes the flavor conversion hard to predict by intuitions. The treatment is linearizing the equation of motion and identify instabilities. One of the tricks in the realm is to utilize the dispersion relation. In principle, dispersion relation tells us how waves propagate for different wave numbers and frequencies. However, the neutrino problems is much more complicated than it. Situations that is inconsistent with the dispersion relation approach are identified.

   Finally, forward scattering of supernova neutrinos are not the only thing happens. During the propagation around supernova, neutrinos may be scattered in every directions which forms a neutrino halo. The halo couples the neutrinos nonlocally, which is a nonlocal boundary value problem. One of the solutions is the relaxation method. Starting from some state of neutrinos and relaxing the system into equilibrium is proven to be a working algorithm. A numerical algorithm is developed and neutrino line model with back scattering is investigated.
\clearpage %(required for 1-page abstract)
\end{abstract}
