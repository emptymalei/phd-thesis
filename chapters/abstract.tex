%!TEX root = ../phd-thesis-lei-ma.tex


\begin{abstract}
   Prediction of the neutrino flavor in astrophysical environments is crucial to many unsolved mysteries such as supernova explosions. Neutrino flavor conversions are altered by the matter, neutrino themselves, as well as the geometries. Neutrinos propagating through matter background experience quantum potential which changes the flavor conversions. Mathematics of such quantum two level systems has been established but without clear pictures. We draw analogies between neutrinos passing through matter and Rabi oscillations in optics which allows us to calculate resonance conditions and flavor survival probability easily. 
   
   As for neutrinos flavors with high number densities, nonlinear interactions come into play. Nonlinearity makes the flavor conversion hard to predict by intuitions. The treatment is linearizing the equation of motion and identify instabilities. One of the tricks in the realm is to utilize the dispersion relation. In principle, dispersion relation tells us how waves propagate for different wave numbers and frequencies. However, the neutrino problems is much more complicated than it. We identify situations that is inconsistent with the dispersion relation approach. 
   
   Finally, neutrinos are scattered during the propagation, which forms a neutrino halo that couples the neutrinos nonlocally. Mathematically this is a nonlocal boundary value problem. One of the solutions is the relaxation method which start from some state of neutrinos and relax the system into equilibrium. We developed the numerical algorithm and investigated the neutrino line model with back scattering.
\clearpage %(required for 1-page abstract)
\end{abstract}
