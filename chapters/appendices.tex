%!TEX root = ../phd-thesis-lei-ma.tex

%%%%%%%%%%%%%%%%%%%%%%%%%%%%%%%%%%%%%%%%%%
%%%%%%%%%%%%% APPENDIX  %%%%%%%%%%%%%%%%%%
%%%%%%%%%%%%%%%%%%%%%%%%%%%%%%%%%%%%%%%%%%


\chapter*{Appendices}
\label{chap:appendices}
\addcontentsline{toc}{chapter}{Appendices}
 % Next lines duplicated from .toc file and used to create mini
 % "Appendix Table of Contents," if desired:
   \contentsline {chapter}{\numberline {A} Rabi Oscillations}{4}
%   \contentsline {chapter}{\numberline {B}Derivation of $A = \pi r^2$}{5}
 % End mini table of contents

\appendix



\chapter{\label{chap:app-sec:conventions}Conventions}

\section{\label{chap:app-sec:conventions-subsec:terms}Terms}

\begin{itemize}
    \item 
    Normal hierarchy for two flavors is always defined as $m_2^2-m_1^2>0$.
    \item
    Solar neutrino mass splitting is $\delta m_{12}$, while atmospheric neutrino mass splitting refers to $\delta m_{23}$.
\end{itemize}




\section{\label{chap:app-sec:conventions-subsec:units}Units}


Natural units system makes the calculations of neutrinos convenient. By definition, we set reduced Planck constant and speed of light to be 1, $\hbar = 1 = c$.
The conversion between natural units and SI can be down by using the following relations,
\begin{align}
   1 \mathrm{GeV} &= 5.08 \times 10^{15} \mathrm {m^{-1}} \\
   1 \mathrm{GeV} &= 1.8\times 10^{-27} \mathrm{kg}
\end{align}
To convert between different physical quantities in this thesis, I always use the following tips.
\begin{itemize}
    \item The energy-mass-momentum relations becomes $E^2 = p^2 + m^2c^2 = p^2 + m^2$. Thus mass $m$, momentum $\mathbf p$ and energy $E$ have the same dimension.
    \item An example of angular momentum in quantum mechanics is $L_z = m\hbar = m$ where $m$ is a quantum number. $\hbar$ is of dimension angular momentum.
    \item A plane wave in quantum mechanics is $\Psi = A e^{ \frac{E t - p x}{\hbar} }$. $\frac{E t - p x}{\hbar}$ should be dimensionless, which means $px$ has dimension angular momentum, which is obvious, meanwhile we notice that $E t$ also has the dimension of angular momentum. Previously we noticed momentum has the same dimension with energy, we should have time $t$ with the same dimension of length $x$. Also we can conclude that length and time have the same dimenson as $1/E$.
\end{itemize}


\section{Useful Conversions in Neutrino Physics}

Using natural units, length = time = 1/energy, we could rescale almost all quantities in neutrino oscillations using energy, or whatever characteristic scale we have.

We use two flavor vacuum oscillations between the two masses $m_1$ and $m_2$ as an example. The characteristic energy scale is the oscillation frequency $\omega_{v,21}$. The equation of motion
\begin{align}
   i\frac{d}{d x} \Psi = \frac{\omega_v}{2}(-\cos 2\theta_v \boldsymbol{\sigma_3} + \sin 2\theta_v \boldsymbol{\sigma_1}) \Psi,
\end{align}
can be rescaled using the characteristic energy scale $\omega_{v,21}$
\begin{align}
   i\frac{d}{d \hat x} \Psi = \frac{1}{2}(-\cos 2\theta_v \boldsymbol{\sigma_3} + \sin 2\theta_v \boldsymbol{\sigma_1})\Psi ,
\end{align}
where $\hat x = \omega_{v,21} x$. It is convenient for numerical calculations to convert quantities into dimensionless ones.

However, we usually discuss oscillation length in SI units for a grip of the picture. To convert from natural units to SI units, we write down the conversion here. The oscillation angular frequency is given by
\begin{align}
   \omega_{v,21} &= \frac{\delta m^2}{2E} \nonumber\\
   &=  \left(\frac{7.5\times 10^{-5}\mathrm{eV}^2}{2\times 1\mathrm{MeV}} \right) \left(\frac{\delta m^2}{7.5\times 10^{-5}\mathrm{eV}^2} \right) \frac{1\mathrm{MeV}}{E} \nonumber\\
   &= 3.75\times 10^{-11}\mathrm{eV}  \left(\frac{\delta m^2}{7.5\times 10^{-5}\mathrm{eV}^2}\right) \left(\frac{1\mathrm{MeV}}{E}\right) .
\end{align}
On the other hand, electro-volt is related to length through the useful formula
\begin{equation}
   197\mathrm{MeV}\cdot \mathrm{fm} = \hbar c = 1.
\end{equation}
Thus we have the oscillation angular frequency written as
\begin{align}
   \omega_{v,21} &= 3.75\times 10^{-11}\mathrm{eV}  \frac{\delta m^2}{7.5\times 10^{-5}\mathrm{eV}^2} \frac{1\mathrm{MeV}}{E} \nonumber\\
   &= 3.75\times 10^{-17}\mathrm{MeV}  \frac{\delta m^2}{7.5\times 10^{-5}\mathrm{eV}^2} \frac{1\mathrm{MeV}}{E}\nonumber \\
   &= 1.90\times 10^{-4}  \mathrm{m}^{-1}  \frac{\delta m^2}{7.5\times 10^{-5}\mathrm{eV}^2} \frac{1\mathrm{MeV}}{E}.
   \label{common-sense-eqn-omega-v-si-unit}
\end{align}
Similarly for $\delta m_{32}=2.4\times 10^{-3}\mathrm{eV^2}$ the frequency is
\begin{align}
   \omega_{v,32} =\frac{\delta m^2_{32}}{2E} = 6.3\times 10^{-3} \mathrm{m}^{-1}  \frac{\delta m^2_{32}}{2.5\times 10^{-3} \mathrm{eV}^2 } \frac{1MeV}{E}.
\end{align}
With the results for angular frequencies, the rescaled length $\hat x$ is restored using
\begin{align}
    x = \frac{\hat x}{\omega_v} &= \frac{\hat x}{  1.90\times 10^{-4}  \mathrm{m}^{-1}  \frac{\delta m^2}{7.5\times 10^{-5}\mathrm{eV}^2} \frac{1\mathrm{MeV}}{E} } \\
    &= \frac{\hat x}{0.190} \mathrm{km} \frac{7.5\times 10^{-5}\mathrm{eV}^2}{\delta m^2}  \frac{E}{1\mathrm{MeV}}.
\end{align}


Another important example is the 2 flavor neutrino oscillations in constant matter background potential $\lambda_c = \sqrt{2}G_{\mathrm F} n_e$. The characteristic energy scale is $\omega_m$ which is calculated using
\begin{equation}
   \omega_m = \omega_v \sqrt{ \frac{\lambda_c}{\omega_v}^2 + 1 - 2 \frac{\lambda_c}{\omega_v}\cos 2\theta_v }.
   \label{common-sense-eqn-omega-m}
\end{equation}
Meanwhile, the effective mixing angle $\theta_m$ is determined by
\begin{equation}
   \tan 2\theta_m = \frac{\sin 2\theta_v}{\cos 2\theta_v - \frac{\lambda}{\omega_v} }.
\end{equation}
Similar to vacuum equation of motion, we rescale the equation of motion in constant background using $\omega_m$
\begin{equation}
   i \frac{d}{d\hat x} \Psi = \frac{1}{2}(-\cos 2\theta_m \boldsymbol{\sigma_3} + \sin 2\theta_m \boldsymbol{\sigma_1})\Psi ,
\end{equation}
we find out the scaled distance
\begin{equation}
   \hat x = \omega_m x.
\end{equation}
To reverse the process and find out the actual SI unit distance after the numerical calculation, we use
\begin{equation}
   x = \frac{\hat x}{\omega_m}.
   \label{common-sense-eqn-actual-distance-constant-matter}
\end{equation}
The procedure will be the following.
\begin{itemize}
    \item Calculate $\omega_v$ using Eq.~\ref{common-sense-eqn-omega-v-si-unit}.
\item Calculate $\hat\lambda_c = \frac{\lambda_c}{\omega_v}$.
\item Calculate $\omega_m$ using Eq.~\ref{common-sense-eqn-omega-m}.
\item Find out the actual distance using Eq.~ \ref{common-sense-eqn-actual-distance-constant-matter}.
\end{itemize}


\chapter{\label{app:rabi-oscillations}Rabi Oscillations}

%  trim={<left> <lower> <right> <upper>}
%\begin{widetext}
% \begin{figure*}
%     \centering
%     \begin{subfigure}[b]{0.5\textwidth}
%         \centering
%         \includegraphics[trim={2cm 3.2cm 9.5cm 1cm},clip]{chapters/assets/rabi/rabi-isospin-static-frame}
%     \caption{}
%     \label{fig-rabi-isospin-static-frame}
%     \end{subfigure}%
%     ~
%     \begin{subfigure}[b]{0.5\textwidth}
%         \centering
%         \includegraphics[trim={6cm 3cm 9.5cm 2cm},clip]{chapters/assets/rabi/rabi-isospin-rotating-frame}
%         \caption{}
%         \label{fig-rabi-isospin-rotating-frame}
%     \end{subfigure}
%     \caption{Rabi oscillations in static frame and rotating frame. In both figures the red dashed vector is the flavor isospin, while the black solid vectors are the vectors of Hamiltonian. The left panel shows the rotating Hamiltonian $\mathbf{H}_3+\mathbf{H}_+$. The right panel shows the rotation of flavor isospin around a static vector $\mathbf{H}'_3+\mathbf{H}'_+$ in the rotating frame.}
%     \label{fig-rabi-isospin-different-frame}
% \end{figure*}
%\end{widetext}

\begin{figure}
        \centering
        \includegraphics[width=\columnwidth, trim={20cm 10cm 50cm 10cm},clip]{chapters/assets/rabi/rabi-isospin-rotating-frame}
    \caption{Rabi oscillations in corotating frame. The red dashed vector is the flavor isospin, while the black solid vectors are the vectors of Hamiltonian. The flavor isospin vector is precessing around vector of total Hamiltonian $\mathbf{H}_3+\mathbf{H}_+$.}
    \label{fig-rabi-isospin-rotating-frame}
\end{figure}

In this appendix we introduce flavor isospin \cite{Duan2006a} to Rabi oscillations and derive the transition probabilities as well as explain the resonance and width briefly.


The Hamiltonian for Rabi oscillation is
\begin{equation}
    H_{\mathrm R} = -\frac{\omega_{\mathrm R}}{2}\sigma_3 - \frac{A_{\mathrm{R}} }{2}  \left( \cos(k_{\mathrm{R}} t +\phi_{\mathrm{R}})\sigma_1  - \sin(k_{\mathrm{R}} t +\phi_{\mathrm{R}}) \sigma_2\right),
    \label{rabi-oscillation-single-perturbation}
\end{equation}
in which $\omega_{\mathrm R}$ serves as the energy split of the two level system, while $A_{\mathrm{R}}$ and $k_{\mathrm{R}}$ are the strength and frequency of the driving field, respectively. A decomposition of the second term shows that
\begin{equation*}
H_{\mathrm R}
= - \frac{\boldsymbol{\sigma}}{2} \cdot (\mathbf{H}_3 + \mathbf{H}_+ ) ,
\end{equation*}
where $\boldsymbol{\sigma}$ is the the vector of Pauli matrices, and the three vectors are
\begin{align}
    \mathbf{H}_3 = & \begin{pmatrix}
    0 \\ 0 \\ \omega_{\mathrm R}
    \end{pmatrix}, \\
    \mathbf{H}_+ = & \begin{pmatrix}
    A_{\mathrm{R}} \cos(k_{\mathrm{R}} t +\phi_{\mathrm{R}}) \\
    - A_{\mathrm{R}} \sin(k_{\mathrm{R}} t +\phi_{\mathrm{R}}) \\
    0
    \end{pmatrix}.
\end{align}

The three vectors are mapped onto a Cartesian coordinate system, so that $\mathbf{H}_3$ is the vector aligned with the third axis, $\mathbf{H}_+$ is a rotating vectors in a plane perpendicular to $\mathbf{H}_3$. The wave function $\Psi=(\psi_1,\psi_2)^{\mathrm{T}}$ is also used to define the state vector $\mathbf{s}$
\begin{align}
    \mathbf{s} =& \Psi^\dagger \frac{\boldsymbol{\sigma}}{2}\Psi \\
    =& \frac{1}{2}\begin{pmatrix}
    2\,\mathrm{Re}\,(\psi_1^* \psi_2) \\
    2\,\mathrm{Im}\,(\psi_1^*\psi_2) \\
    \lvert \psi_1 \rvert^2 - \lvert \psi_2 \rvert^2
    \end{pmatrix}
\end{align}
The third component of $\mathbf{s}$, which is denoted as $s_3$, is within range $[-1/2,1/2]$. The two limits, $s_3=-1/2$ and $s_3=1/2$ stand for the system in high energy state and low energy state respectively. $s_3=0$ if the system has equal probabilities to be on high energy state and low energy state. The Schr\"odinger equation becomes
\begin{equation}
\frac{\mathrm{d}}{\mathrm{d} t } \mathbf{s} = \mathbf{s} \times \mathbf{H},
\end{equation}
which is the precession equation. For static $\mathbf{H}$, the state vector $\mathbf{s}$ precess around $\mathbf{H}$.

In a frame that corotates with $\mathbf{H}_+$, which is described in Fig.~\ref{fig-rabi-isospin-rotating-frame}, the new Hamiltonian is
\begin{equation}
\frac{\mathrm d}{\mathrm d t } \mathbf{s} = \mathbf{s} \times (\mathbf{H}'_3 + \mathbf{H}‘_+),
\end{equation}
where
\begin{equation}
\mathbf{H}'_3 = \begin{pmatrix}
    0 \\ 0 \\  \omega_{\mathrm{R}} - k_{\mathrm R}
    \end{pmatrix}, \qquad \mathbf{H}'_+ = \begin{pmatrix}
    A_{\mathrm{R}} \\ 0 \\  0
    \end{pmatrix}.
\end{equation}
The state vector $\mathbf{s}$ precess around a static vector $\mathbf{H}'_3 + \mathbf{H}'_+$ with a frequency $\Omega_{\mathrm R} = \sqrt{ \lvert A_{\mathrm{R}}\rvert^2 + (k_{\mathrm{R}} - \omega_{\mathrm R})^2 }$. A geometric analysis by projecting the state vector $\mathbf{s}$ on to the verticle axis shows that
\begin{equation}
s_3 = \frac{1}{2} - \frac{\lvert A_{\mathrm R}\rvert ^2}{\Omega_{\mathrm R}^2}\sin^2\left(\frac{\Omega_{\mathrm R}}{2} t\right).
\end{equation}
Resonance occurs when the term $\mathbf{H}_3$ disappears in this corotating frame, since the state vector $\mathbf{s}$ converts completely between $+1/2$ (low energy state) and $-1/2$ (high energy state).



Such a system has analytical transition probability from low energy state to high energy state
\begin{equation}
    P(t) = \frac{1}{2}(1- 2 s_3(t))= \frac{\left \lvert A_{\mathrm{R}} \right \rvert ^2}{ \Omega_{\mathrm R}^2 } \sin^2 \left( \frac{\Omega_{\mathrm R}}{2} t \right),
    \label{rabi-system-transition-probability}
\end{equation}
where
\begin{equation}
\Omega_{\mathrm R} = \sqrt{ \lvert A_{\mathrm{R}}\rvert^2 + (k_{\mathrm{R}} - \omega_{\mathrm R})^2 }
\end{equation} is known as Rabi frequency. The detuning, which is defined by $k_{\mathrm{R}} - \omega_{\mathrm R}$, determines how off-resonance the system is, and amplitude of driving field $A_{\mathrm{R}}$ determines the resonance width,
\begin{align}
\text{Detuning} =&~\lvert k_{\mathrm{R}} - \omega_{\mathrm R} \rvert, \\
\text{Resonance Width} =&~\lvert A_{\mathrm R} \rvert.
\end{align}
The transition probability oscillates with frequency $\Omega_{\mathrm R}$. However, the amplitude $A_1$ is the dominate factor for oscillation frequency when the system is close to resonance. The phase of the matter potential $\phi_{\mathrm{R}}$ has no effect on the transition probability since it only determines the initial phase of driving Hamiltonian vector $\mathbf{H}_+$. We also notice that the transition amplitude is determined by relative detuning $\RD$, which is defined as
\begin{equation}
    \RD = \left\lvert \frac{ k_{\mathrm R} - \omega_{\mathrm R}}{A_{\mathrm R}} \right\rvert.
\end{equation}


Given a Rabi oscillation system with two driving frequencies
\begin{align*}
    H_{\mathrm R} =& -\frac{\omega_{\mathrm R}}{2}\sigma_3 - \frac{A_{1} }{2}  \left( \cos(k_{1} t +\phi_{1})\sigma_1  - \sin(k_{1} t +\phi_{1}) \sigma_2\right) \nonumber\\
    & - \frac{A_{2} }{2}  \left( \cos(k_{2} t +\phi_{2})\sigma_1  - \sin(k_{2} t +\phi_{2}) \sigma_2\right)
\end{align*}
we decompose it into $\mathbf{H}_{\mathrm R}=\mathbf{H}_3 + \mathbf{H}_{1} + \mathbf{H}_2$, where
\begin{equation*}
    \mathbf{H}_1 =  \begin{pmatrix}
     A_{1} \cos(k_{1}t+\phi_{1}) \\
     A_{1} \sin(k_{1}t+\phi_{1})  \\
     0
      \end{pmatrix},   \mathbf{H}_2 =  \begin{pmatrix}
     A_{2} \cos(k_{2}t+\phi_{2}) \\
     A_{2} \sin(k_{2}t+\phi_{2})  \\
     0
      \end{pmatrix}.
\end{equation*}
Assume $\mathbf{H}_1$ is the on-resonance perturbation which requires $k_1 = \omega_{\mathrm{R}}$. The most general condition that we can drop the new perturbation $\mathbf{H}_2$ is to make sure $k_2$ is far from the resonance condition compared to the resonance width,
\begin{equation}
\RD \equiv \frac{\lvert k_2 -\omega_{\mathrm R}\rvert}{\lvert A_2\rvert} \gg 1.
\end{equation}

The transition amplitude between the two states becomes
\begin{equation}
P(t) = \frac{1}{1+\RD^2}\sin^2(\frac{\Omega_{\mathrm{R}}}{2}t).
\end{equation}


