%!TEX root = ../phd-thesis-lei-ma.tex

%%%%%%%%%%%%%%%%%%%%%%%%%%%%%%%%%%%%%%%%%%
%%%%%%%%%%%%% APPENDIX  %%%%%%%%%%%%%%%%%%
%%%%%%%%%%%%%%%%%%%%%%%%%%%%%%%%%%%%%%%%%%


\chapter*{Appendices}
\label{chap:appendices}
\addcontentsline{toc}{chapter}{Appendices}
 % Next lines duplicated from .toc file and used to create mini
 % "Appendix Table of Contents," if desired:
   \contentsline {chapter}{\numberline {A} Rabi Oscillations}{4}
%   \contentsline {chapter}{\numberline {B}Derivation of $A = \pi r^2$}{5}
 % End mini table of contents

\appendix


\chapter{\label{app:rabi-oscillations}Rabi Oscillations}

%  trim={<left> <lower> <right> <upper>}
%\begin{widetext}
% \begin{figure*}
%     \centering
%     \begin{subfigure}[b]{0.5\textwidth}
%         \centering
%         \includegraphics[trim={2cm 3.2cm 9.5cm 1cm},clip]{chapters/assets/rabi/rabi-isospin-static-frame}
%     \caption{}
%     \label{fig-rabi-isospin-static-frame}
%     \end{subfigure}%
%     ~
%     \begin{subfigure}[b]{0.5\textwidth}
%         \centering
%         \includegraphics[trim={6cm 3cm 9.5cm 2cm},clip]{chapters/assets/rabi/rabi-isospin-rotating-frame}
%         \caption{}
%         \label{fig-rabi-isospin-rotating-frame}
%     \end{subfigure}
%     \caption{Rabi oscillations in static frame and rotating frame. In both figures the red dashed vector is the flavor isospin, while the black solid vectors are the vectors of Hamiltonian. The left panel shows the rotating Hamiltonian $\mathbf{H}_3+\mathbf{H}_+$. The right panel shows the rotation of flavor isospin around a static vector $\mathbf{H}'_3+\mathbf{H}'_+$ in the rotating frame.}
%     \label{fig-rabi-isospin-different-frame}
% \end{figure*}
%\end{widetext}

\begin{figure}
        \centering
        \includegraphics[width=\columnwidth, trim={20cm 10cm 50cm 10cm},clip]{chapters/assets/rabi/rabi-isospin-rotating-frame}
    \caption{Rabi oscillations in corotating frame. The red dashed vector is the flavor isospin, while the black solid vectors are the vectors of Hamiltonian. The flavor isospin vector is precessing around vector of total Hamiltonian $\mathbf{H}_3+\mathbf{H}_+$.}
    \label{fig-rabi-isospin-rotating-frame}
\end{figure}

In this appendix we introduce flavor isospin \cite{Duan2006a} to Rabi oscillations and derive the transition probabilities as well as explain the resonance and width briefly.


The Hamiltonian for Rabi oscillation is
\begin{equation}
    H_{\mathrm R} = -\frac{\omega_{\mathrm R}}{2}\sigma_3 - \frac{A_{\mathrm{R}} }{2}  \left( \cos(k_{\mathrm{R}} t +\phi_{\mathrm{R}})\sigma_1  - \sin(k_{\mathrm{R}} t +\phi_{\mathrm{R}}) \sigma_2\right),
    \label{rabi-oscillation-single-perturbation}
\end{equation}
in which $\omega_{\mathrm R}$ serves as the energy split of the two level system, while $A_{\mathrm{R}}$ and $k_{\mathrm{R}}$ are the strength and frequency of the driving field, respectively. A decomposition of the second term shows that
\begin{equation*}
H_{\mathrm R}
= - \frac{\boldsymbol{\sigma}}{2} \cdot (\mathbf{H}_3 + \mathbf{H}_+ ) ,
\end{equation*}
where $\boldsymbol{\sigma}$ is the the vector of Pauli matrices, and the three vectors are
\begin{align}
    \mathbf{H}_3 = & \begin{pmatrix}
    0 \\ 0 \\ \omega_{\mathrm R}
    \end{pmatrix}, \\
    \mathbf{H}_+ = & \begin{pmatrix}
    A_{\mathrm{R}} \cos(k_{\mathrm{R}} t +\phi_{\mathrm{R}}) \\
    - A_{\mathrm{R}} \sin(k_{\mathrm{R}} t +\phi_{\mathrm{R}}) \\
    0
    \end{pmatrix}.
\end{align}

The three vectors are mapped onto a Cartesian coordinate system, so that $\mathbf{H}_3$ is the vector aligned with the third axis, $\mathbf{H}_+$ is a rotating vectors in a plane perpendicular to $\mathbf{H}_3$. The wave function $\Psi=(\psi_1,\psi_2)^{\mathrm{T}}$ is also used to define the state vector $\mathbf{s}$
\begin{align}
    \mathbf{s} =& \Psi^\dagger \frac{\boldsymbol{\sigma}}{2}\Psi \\
    =& \frac{1}{2}\begin{pmatrix}
    2\,\mathrm{Re}\,(\psi_1^* \psi_2) \\
    2\,\mathrm{Im}\,(\psi_1^*\psi_2) \\
    \lvert \psi_1 \rvert^2 - \lvert \psi_2 \rvert^2
    \end{pmatrix}
\end{align}
The third component of $\mathbf{s}$, which is denoted as $s_3$, is within range $[-1/2,1/2]$. The two limits, $s_3=-1/2$ and $s_3=1/2$ stand for the system in high energy state and low energy state respectively. $s_3=0$ if the system has equal probabilities to be on high energy state and low energy state. The Schr\"odinger equation becomes
\begin{equation}
\frac{\mathrm{d}}{\mathrm{d} t } \mathbf{s} = \mathbf{s} \times \mathbf{H},
\end{equation}
which is the precession equation. For static $\mathbf{H}$, the state vector $\mathbf{s}$ precess around $\mathbf{H}$.

In a frame that corotates with $\mathbf{H}_+$, which is described in Fig.~\ref{fig-rabi-isospin-rotating-frame}, the new Hamiltonian is
\begin{equation}
\frac{\mathrm d}{\mathrm d t } \mathbf{s} = \mathbf{s} \times (\mathbf{H}'_3 + \mathbf{H}‘_+),
\end{equation}
where
\begin{equation}
\mathbf{H}'_3 = \begin{pmatrix}
    0 \\ 0 \\  \omega_{\mathrm{R}} - k_{\mathrm R}
    \end{pmatrix}, \qquad \mathbf{H}'_+ = \begin{pmatrix}
    A_{\mathrm{R}} \\ 0 \\  0
    \end{pmatrix}.
\end{equation}
The state vector $\mathbf{s}$ precess around a static vector $\mathbf{H}'_3 + \mathbf{H}'_+$ with a frequency $\Omega_{\mathrm R} = \sqrt{ \lvert A_{\mathrm{R}}\rvert^2 + (k_{\mathrm{R}} - \omega_{\mathrm R})^2 }$. A geometric analysis by projecting the state vector $\mathbf{s}$ on to the verticle axis shows that
\begin{equation}
s_3 = \frac{1}{2} - \frac{\lvert A_{\mathrm R}\rvert ^2}{\Omega_{\mathrm R}^2}\sin^2\left(\frac{\Omega_{\mathrm R}}{2} t\right).
\end{equation}
Resonance occurs when the term $\mathbf{H}_3$ disappears in this corotating frame, since the state vector $\mathbf{s}$ converts completely between $+1/2$ (low energy state) and $-1/2$ (high energy state).



Such a system has analytical transition probability from low energy state to high energy state
\begin{equation}
    P(t) = \frac{1}{2}(1- 2 s_3(t))= \frac{\left \lvert A_{\mathrm{R}} \right \rvert ^2}{ \Omega_{\mathrm R}^2 } \sin^2 \left( \frac{\Omega_{\mathrm R}}{2} t \right),
    \label{rabi-system-transition-probability}
\end{equation}
where
\begin{equation}
\Omega_{\mathrm R} = \sqrt{ \lvert A_{\mathrm{R}}\rvert^2 + (k_{\mathrm{R}} - \omega_{\mathrm R})^2 }
\end{equation} is known as Rabi frequency. The detuning, which is defined by $k_{\mathrm{R}} - \omega_{\mathrm R}$, determines how off-resonance the system is, and amplitude of driving field $A_{\mathrm{R}}$ determines the resonance width,
\begin{align}
\text{Detuning} =&~\lvert k_{\mathrm{R}} - \omega_{\mathrm R} \rvert, \\
\text{Resonance Width} =&~\lvert A_{\mathrm R} \rvert.
\end{align}
The transition probability oscillates with frequency $\Omega_{\mathrm R}$. However, the amplitude $A_1$ is the dominate factor for oscillation frequency when the system is close to resonance. The phase of the matter potential $\phi_{\mathrm{R}}$ has no effect on the transition probability since it only determines the initial phase of driving Hamiltonian vector $\mathbf{H}_+$. We also notice that the transition amplitude is determined by relative detuning $\RD$, which is defined as
\begin{equation}
    \RD = \left\lvert \frac{ k_{\mathrm R} - \omega_{\mathrm R}}{A_{\mathrm R}} \right\rvert.
\end{equation}


Given a Rabi oscillation system with two driving frequencies
\begin{align*}
    H_{\mathrm R} =& -\frac{\omega_{\mathrm R}}{2}\sigma_3 - \frac{A_{1} }{2}  \left( \cos(k_{1} t +\phi_{1})\sigma_1  - \sin(k_{1} t +\phi_{1}) \sigma_2\right) \nonumber\\
    & - \frac{A_{2} }{2}  \left( \cos(k_{2} t +\phi_{2})\sigma_1  - \sin(k_{2} t +\phi_{2}) \sigma_2\right)
\end{align*}
we decompose it into $\mathbf{H}_{\mathrm R}=\mathbf{H}_3 + \mathbf{H}_{1} + \mathbf{H}_2$, where
\begin{equation*}
    \mathbf{H}_1 =  \begin{pmatrix}
     A_{1} \cos(k_{1}t+\phi_{1}) \\
     A_{1} \sin(k_{1}t+\phi_{1})  \\
     0
      \end{pmatrix},   \mathbf{H}_2 =  \begin{pmatrix}
     A_{2} \cos(k_{2}t+\phi_{2}) \\
     A_{2} \sin(k_{2}t+\phi_{2})  \\
     0
      \end{pmatrix}.
\end{equation*}
Assume $\mathbf{H}_1$ is the on-resonance perturbation which requires $k_1 = \omega_{\mathrm{R}}$. The most general condition that we can drop the new perturbation $\mathbf{H}_2$ is to make sure $k_2$ is far from the resonance condition compared to the resonance width,
\begin{equation}
\RD \equiv \frac{\lvert k_2 -\omega_{\mathrm R}\rvert}{\lvert A_2\rvert} \gg 1.
\end{equation}

The transition amplitude between the two states becomes
\begin{equation}
P(t) = \frac{1}{1+\RD^2}\sin^2(\frac{\Omega_{\mathrm{R}}}{2}t).
\end{equation}


