%%%%%%%%%%%%%%%%%%%%%%%%%%%%%%%%%%%
%%%%%%%%% Biblatex %%%%%%%%%%%%%%%
\usepackage[
backend=biber,
style=phys,
sorting=ynt
]{biblatex}

\AtEveryBibitem{%
  \clearfield{issn} % Remove issn
  \clearfield{doi} % Remove doi

  \ifentrytype{online}{}{% Remove url except for @online
    \clearfield{url}
  }
}

% \addcontentsline{toc}{chapter}{References}
% \bibliographystyle{apalike2}
% \bibliographystyle{abbrv}
\addbibresource{references.bib}

% \DeclareLabelname[movie]{
%   \field{director}
%   \field{producer}
% }


% \usepackage{natbib}

%%%%%%%%% Biblatex %%%%%%%%%%%%%%%
%%%%%%%%%%%%%%%%%%%%%%%%%%%%%%%%%%%



\usepackage{graphicx}% Include figure files
% \graphicspath{ {chapters/assets/} }

%\usepackage{subcaption}
%\usepackage[caption=false]{subfig}

\usepackage{dcolumn}% Align table columns on decimal point
\usepackage{bm}% bold math
%\usepackage{hyperref}% add hypertext capabilities
%\usepackage[mathlines]{lineno}% Enable numbering of text and display math
%\linenumbers\relax % Commence numbering lines

%\usepackage[showframe,%Uncomment any one of the following lines to test
%%scale=0.7, marginratio={1:1, 2:3}, ignoreall,% default settings
%%text={7in,10in},centering,
%%margin=1.5in,
%%total={6.5in,8.75in}, top=1.2in, left=0.9in, includefoot,
%%height=10in,a5paper,hmargin={3cm,0.8in},
%]{geometry}

% \usepackage{amsmath}
\usepackage{amsmath,thmtools}
\usepackage{amssymb}

\usepackage{tikz}
\usepackage{adjustbox}
\usepackage{color}
\usepackage{xcolor}

\usepackage[normalem]{ulem}
\usepackage{accents}
\usepackage{mathrsfs}

\usepackage{caption}
\usepackage{subcaption}

\usepackage{lineno}
% \linenumbers

\usepackage{hyperref}

\overfullrule=0pt

%%%%%%%%%%%%%%%%%%%%%%%%%%%
%%%%%%  PREAMBLES %%%%%%%%%
\newcommand{\ud}[1]{{#1^{\dagger}}}
\newcommand{\bra}[1]{\left\langle #1\right|}
\newcommand{\ket}[1]{\left| #1\right\rangle}
\newcommand\Tr{\mathrm{Tr}}
\newcommand{\braket}[2]{\langle #1 \mid #2 \rangle}
\newcommand\I{\mathbb{I}}
\newcommand{\avg}[1]{\left< #1 \right>}
\newcommand{\sech}[1]{{\operatorname{sech}{#1}}}
\newcommand{\csch}[1]{{\operatorname{csch}{#1}}}
\newcommand{\RD}{D}
\newcommand{\ri}{\mathrm{i}}
\DeclareMathOperator{\sign}{sign}




\newcommand{\ii}{\mathrm i}
\newcommand{\vv}{\mathrm v}
\newcommand{\ff}{\mathrm f}
\newcommand{\mm}{\mathrm m}
\newcommand{\ee}{\mathrm e}
\newcommand{\xx}{\mathrm x}
\newcommand{\RR}{\mathrm R}
\newcommand{\dd}{\mathrm d}
\newcommand{\FF}{\mathrm F}
\newcommand{\BB}{\mathrm F}

\newcommand{\vph}{v_{\mathrm{ph}}}

\usepackage[]{algorithm2e}
% \usepackage{algorithm}
% \usepackage[noend]{algpseudocode}

%%%%%%  PREAMBLES %%%%%%%%%
%%%%%%%%%%%%%%%%%%%%%%%%%%%
